\section{Budget and sustainability}

\subsection{Sustainability auto-evaluation}

Nowadays sustainability is a great issue, the progress made by humanity in the technology
field it has some consequences too. Information and Communication Technologies are not 
an exception although it has taken some time to gain importance. The appearance of 
small devices like smartphones and the huge power requirements of big data centers have 
helped in gaining conscience of power consumption. 

When trying to build a new project now it's important to see project's sustainability. If the
project involves building some hardware then it may have a negative impact on the environment
so proper measures need to be taken to avoid future problems.

Moreover, if the project it's software related then it has some implications too. In general
when working with this type of projects the only part where it can be a problem is regarding
power consumption, which is usually related to efficiency. As a general rule they should
avoid using too much power but it depends on the project's type. Machine Learning ones
usually have a big power consumption during the training phase since it requires a huge
amount of computational power. However, smartphone related applications tend to have less
consumption to allow for a longer battery life.

However, regarding the budget, solutions that tend to be more sustainable usually are more
expensive. When it is attempted to reduce consumption in software projects, this usually 
translates in more programmer hours and thus a higher budget.

\subsection{Project budget \label{sec:budget}}

In order to carry out this project some resources will be needed. In this document, 
an estimation of the cost of the project is presented, keeping in mind the required
hardware and software resources, and its corresponding amortizations. Also, the
indirect costs of the project are counted too.

\subsubsection{Hardware budget}

In order to implement the different models previously explained, some hardware will be
needed. The training will usually be done in Compute Canada computing platform. The
access to the different provided clusters it's provided by the Bioinformatics and 
Computational Genomics Laboratory.

\autoref{tab:hardware-budget} shows an estimation of the cost of the hardware used in the
project, taking into account its useful life and its amortizations.

\begin{table}[H]
  \centering
  \begin{tabular}{|l|r|r|r|r|}
    \hline
    \textbf{Product} & \textbf{Price} & \textbf{Units} & \textbf{Useful life} 
    & \textbf{Amortization} \\ \hline\hline

    MSI GL62M 7RD-429XES & 1.000 € & 1 & 5 years & 83 € \\ \hline
    Lab iMac & 0 € & 1 & -- & 0 € \\ \hline
    Compute Canada platform & 0 € & 1 & -- & 0 € \\ \hline

    \hline\hline
    \textbf{Total} & 1.000 € & \multicolumn{2}{|c|}{} & 83 € \\ \hline
  \end{tabular}
  \caption{Hardware budget \label{tab:hardware-budget}}
\end{table}

\subsubsection{Software budget}

Some software will be used to carry out the project. Since this project will be used using
Open Source and free tools the total software cost of the project is zero. 
\autoref{tab:software-budget} shows an estimation of the cost of the software used in the 
project.

\begin{table}[H]
  \centering
  \begin{tabular}{|l|r|r|r|}
    \hline
    \textbf{Product} & \textbf{Price} & \textbf{Units} & \textbf{Amortization} \\ \hline\hline

    Python & 0 € & 1 & 0 € \\ \hline
    Tensorflow & 0 € & 1 & 0 € \\ \hline
    PyDicom & 0 € & 1 & 0 € \\ \hline
    NumPy & 0 € & 1 & 0 € \\ \hline
    Scikit & 0 € & 1 & 0 € \\ \hline
    Visual Studio Code & 0 € & 1 & 0 € \\ \hline
    Git & 0 € & 1 & 0 € \\ \hline
    GitHub & 0 € & 1 & 0 € \\ \hline
    PyCharm Community Edition & 0 € & 1 & 0 € \\ \hline
    \LaTeX & 0 € & 1 & 0 € \\ \hline
    Vim & 0 € & 1 & 0 € \\ \hline

    \hline\hline
    \textbf{Total} & 0 € &  & 0  € \\ \hline
  \end{tabular}
  \caption{Software budget \label{tab:software-budget}}
\end{table}


\subsubsection{Human resources budget}

To develop this software some roles will be taken. There will be the Project Manager role,
the Software Developer role and the Tester role. Each role is differentiated in the total
of 760 hours. \autoref{tab:salary} shows the estimated costs for each role and the
Human Resources cost, \autoref{tab:time-estimation} shows an estimation of the time 
inverted on each role for each task.

\begin{table}[H]
  \centering
  \begin{tabular}{|l|r|r|r|}
    \hline
    \textbf{Role} & \textbf{Hours} & \textbf{€/hour} & \textbf{Salary} \\ \hline\hline

    Project Manager & 70 & 33,8 & 2.366 € \\ \hline
    Software Developer & 510 & 23,4 & 11.914 € \\ \hline
    Tester & 180 & 19,5 & 3.510 € \\ \hline

    \hline\hline 
    Total & 760 & & 17.810 € \\
    \hline
  \end{tabular}

  \caption{Human resources budget, salary according to \cite{ine:salary} \label{tab:salary}}
\end{table}

\begin{table}[H]
  \centering
  \begin{tabular}{|P{4cm}|r|r|r|r|}
    \hline
    \multirow{2}{*}[-1em]{\textbf{Task}} & 
    \multirow{2}{*}[-1em]{\textbf{Duration}} & 
    \multicolumn{3}{|c|}{\textbf{Dedication}} \\ \cline{3-5}

     & & \parbox[c][1.5cm]{2.1cm}{\textbf{Project \\ Manager}} & 
     \parbox[c][1.5cm]{2.2cm}{\textbf{Software \\ Developer}} & 
     \textbf{Tester} \\ \hline\hline

     Acquire background in CNN & 120 & 10 & 110 & 0 \\ \hline
     Get familiar with survival models & 80 & 10 & 70 & 0 \\ \hline
     Preprocess data & 40 & 10 & 20 & 10 \\ \hline
     Build shallow siamese network & 200 & 15 & 110 & 75 \\ \hline
     Build deep siamese network & 80 & 10 & 50 & 20 \\ \hline
     Compare models & 80 & 10 & 40 & 30 \\ \hline
     Final stage & 160 & 5 & 110 & 45 \\ 

     \hline\hline
     \textbf{Total} & 760 & 70 & 510 & 180 \\
     \hline
  \end{tabular}

  \caption{Time estimation by role \label{tab:time-estimation}}
\end{table}

\subsubsection{Unexpected costs}

As there may be unexpected changes in the project, an extra budget has to be added to compensate
for such problems. \autoref{tab:unexpected-costs} shows the distribution of this budget per role.

\begin{table}[H]
  \centering
  \begin{tabular}{|l|r|r|r|}
    \hline
    \textbf{Role} & \textbf{Hours} & \textbf{€ / hour} & \textbf{Salary} \\ \hline\hline
    Project Manager & 10 & 33.8 & 338 € \\ \hline
    Software Developer & 20 & 23.4 & 468 € \\ \hline
    Tester & 10 & 19.5 & 195 € \\ \hline

    \hline\hline
    \textbf{Total} & 40 & & 1001 € \\ 
    \hline
  \end{tabular}
  \caption{Unexpected costs \label{tab:unexpected-costs}}
\end{table}

\subsubsection{Indirect costs}

In a project, there are some costs that cannot be added in neither of the previous categories,
this are estimated in \autoref{tab:indirect-costs}.

\begin{table}[H]
  \centering
  \begin{tabular}{|l|r|r|r|}
    \hline
    \textbf{Product} & \textbf{Price} & \textbf{Units} & \textbf{Cost} \\ \hline\hline

    Electricity & 0,12 €/kWh & 550 kWh & 66 € \\ \hline
    Internet + phone & 70 €/month & 4 months & 280 € \\ \hline
    
    \hline\hline
    \textbf{Total} & & & 346 € \\ \hline
  \end{tabular}
  \caption{Indirect costs \label{tab:indirect-costs}}
\end{table}

\subsubsection{Total budget}

By adding all the previously provided budgets, the total estimated budget for this project
is obtained, as shown in table \autoref{tab:total-costs}. Notice that a 5\% of contingency
has been added over the cumulative total in order to cover unexpected expenses that may 
occur during the development of the project.

\begin{table}[H]
  \centering
  \begin{tabular}{|l|r|}
    \hline
    \textbf{Concept} & \textbf{Estimated Costs} \\ \hline\hline

    Hardware resources & 83 € \\ \hline
    Software resources & 0 € \\ \hline
    Human resources & 17.810 € \\ \hline
    Unexpected costs & 1001 € \\ \hline
    Indirect costs & 346 € \\ \hline

    \hline\hline
    \textbf{Subtotal} & 19.240 € \\
    \hline\hline
    Contingency (5\%) & 962 € \\
    \hline\hline
    \textbf{Total} & \textbf{20.202 €} \\ \hline
  \end{tabular}

  \caption{Total project costs \label{tab:total-costs}}
\end{table}

Finally, an estimation of the approximate budget per task is also provided in table 9. The
cost of each task is computed using the number of hours spent in the task and the required
resources.

\begin{table}[H]
  \centering
  \begin{tabular}{|l|r|}
    \hline
    \textbf{Task} & \textbf{Estimated cost} \\ 
    \hline\hline

    Acquire background in CNN & 3.189,79 € \\ \hline
    Get familiar with survival models & 2.126,53 € \\ \hline
    Preprocess data & 1.063,26 € \\ \hline
    Build shallow siamese network & 5.316,32 € \\ \hline
    Build deep siamese network & 2.126,53 € \\ \hline
    Compare models & 2.126,53 € \\ \hline
    Final stage & 4,253.05 € \\ 

    \hline\hline
    \textbf{Total} & 14.899,00 € \\ \hline
  \end{tabular}
\end{table}


\subsection{Budget control}

Controlling the spent budget is a very important part of the project. Using too much time or 
having to buy unexpected hardware could be a problem but it's likely to happen. That's why
some plans will be made to avoid, as much as possible, this situation.

It may be necessary to buy extra software or to have more human resources although it's
unlikely that it may be necessary more hardware. Solving the extra software requirement
shouldn't be a big problem as there are many free or open source software solutions in
the market.

To control the human resources budget, the number of hours working as each role will be 
recorded, this way then it can be seen if too much time has been used for a task and then
it can be solved by rescheduling the next tasks.

\subsection{Sustainability and social commitment}

In order to evaluate the project's sustainability, environmental impact will be analyzed
regarding three major factors: environmental, social and economical. The analysis will be 
based in the application of the sustainability matrix to the project shown in
\autoref{tab:sustainability} and in the poll that can be accessed through 
\url{https://goo.gl/kWLMLE}.

\begin{table}[H]
  \centering
  \begin{tabular}{|c|c|c|c|}
    \cline{2-4}
    \multicolumn{1}{c|}{} & \textbf{PPP} & \textbf{Useful life} & \textbf{Risks} \\ 
    \hhline{-===}

    \multirow[c]{2}{*}{\textbf{Environmental}} & 
    \makecell{Design \\ consumption} & \makecell{Ecological \\ footprint} & 
    \makecell{Environmental \\ risks} \\ \cline{2-4}
    & 2/10 & 19/20 & -2/-20 \\ \hline
    
    \multirow{2}{*}{\textbf{Economical}} & 
    Bill & \makecell{Viability \\ plan} & \makecell{Economical \\ risks} \\ \cline{2-4}
    & 7/10 & 18/20 & -15/-20 \\ \hline

    \multirow{2}{*}{\textbf{Social}} &
    \makecell{Personal \\ impact} & \makecell{Social \\ Impact} & 
    \makecell{Social \\ risks} \\ \cline{2-4} 
    & 9/10 & 18/20 & -5/-20 \\ \hline

    \hline\hline
    \multirow{2}{*}{\parbox[c]{3cm}{\centering\textbf{Sustainability \\ range}}} &
    18/30 & 55/60 & -22/-60 \\ \cline{2-4}
    & \multicolumn{3}{c|}{51/90} \\ \hline

  \end{tabular}
  \caption{Sustainability matrix of the project \label{tab:sustainability}}
\end{table}

\subsubsection{Environmental dimension}

Regarding the environmental dimension the project has two main stages. Since the main objective
of the project is to develop a deep learning model the first stage is to create the model
itself. The second one is to deploy the product and start using it.

In the first stage many resources will be needed since the training process in machine learning
requires a huge amount of computational power. This is directly translated to electricity 
consumption so, during the development process the environmental impact will be hight.

However, once a model is already trained, using it does not require much power since the
data inference it's quite fast and thus, does not require many electricity consumption.
That's why the second part of the project won't have a very hight environmental impact.

\subsubsection{Social dimension}

As there is a lot of research regarding this topic, creating this project should help pushing 
forward the development of new methods for survival analysis. Either if the project results 
are positive or negative it should help to prove that a certain method is valid or not. 

In addition, it's primary focus is medical, so it's very social oriented. Thus,
in the long term, it should help cancer patients to receive better treatments 
that will be developed with help of survival analysis models. 

Regarding the personal and ethical implications, it will help having a better understanding
of machine learning models, how to use them and how to develop them. Moreover, this project
will not have any ethical issue but the contrary, knowing that the social outcome can be
positive it's an incentive to work and do a good job.

\subsubsection{Economical dimension}

A detailed exposition of all the related project costs has already been made including both
human and material resources, as shown in \autoref{sec:budget}.

This project should have a positive impact regarding the economical dimension. Improving
the methods for survival detection helps reducing the costs of future research. However, 
there's the downside that if the research has some kind of mistake that it's not noticed,
a negative impact may occur due to the fact that some related researches may have to roll
back some results.
