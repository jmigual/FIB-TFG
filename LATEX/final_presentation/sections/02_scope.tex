% !TEX root = ../main.tex

\section{Project scope}
\subsection{Problem formulation}
\begin{frame}{\insertsubsec}
  \begin{itemize}
    \item We have access to a unique set of \( \simeq 500 \) scans
    \item Develop a new deep learning model to analyze this dataset
    \item Get better results than the \emph{volume} radiomic feature which usually achieves
    a C-index of 0.65 for the Head and Neck dataset. 
  \end{itemize}
\end{frame}

\subsection{State-of-the-art}
\begin{frame}{\insertsubsec}
  The most typical approach is to extract radiomic features, usually with
  the \emph{PyRadiomics} package, from the MRI, PET or CT scans.

  \vspace{.5cm}
  An alternative approach is to use deep-learning based models for prediction or 
  feature extraction. Pre-trained models have reduced the requirements for big data sets.
  Possible strategies are:
  \begin{itemize}
    \item Use a pre-trained CNN as a feature extractor
    \item Fine tune a pre-trained CNN on medical data
  \end{itemize}
  
\end{frame}

\subsection{Stakeholders}
\begin{frame}{\insertsubsec}
  \begin{itemize}
    \item Developer: responsible for research, document and implement the software
    \item Director: responsible for guiding, giving advice and helping the Developer
    \item Beneficiaries: future researchers or patients depending on the outcome
  \end{itemize}
\end{frame}

\subsection{Methodology}
\begin{frame}{\insertsubsec}
  \begin{itemize}
    \item As a research project it will have a process of trial and error
    \item Once in a while the project will be presented in the lab weekly meetings
    \item Every week a meeting with the Principal Investigator will be scheduled
  \end{itemize}
\end{frame}

\subsection{Obstacles}
\begin{frame}{\insertsubsec}
  Some obstacles may be found during the project:

  \begin{itemize}
    \item Training time
    \item Bugs
    \item Scheduling
    \item Not enough data
  \end{itemize}
\end{frame}
