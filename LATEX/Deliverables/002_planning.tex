% !TEX root = main.tex


\section{Project Planning}

\subsection{Planning and scheduling}

The estimated project duration is of about 4 months. The project starts on Wednesday 14th of 
February, 2018 and the deadline is on Sunday 17th June, 2018, the week before the 
presentations start.

During the development of the project there will be weekly lab meetings with my Principal
Investigator, prof. Benjamin Haibe-Kains, where the development of the project will be 
discussed. There I should show my work done and how to approach each of the problems as
they appear.

Every week there's a lab meeting. Its objective is to show the development and improvements 
of different lab members and receive feedback. Once in a while I will be showing work
done there.

It must be noticed that the initial planning can be revised and updated as a result of the 
project's evolution, feedback received from the lab members and from my Principal Investigator. 

\subsection{Task description}

\subsubsection{Acquire background in Convolutional Neural Networks}

The first step is to acquire a better understanding in how a convolutional neural network works.
Therefore, in the las month I've been learning about Convolutional Neural Networks and how they
can be used. 
I started with basic statistics applied to \emph{Machine Learning} by reading the book 
\emph{The Elements of Statistical Learning}.~\cite{neural:elements-statistical-learning}

Then, I continued by doing three courses made by \href{https://www.deeplearning.ai}{Deeplearning.ai}
and published at Coursera~\cite{neural:coursera} related to Convolutional Neural Networks:
\begin{itemize}
  \item \href{https://www.coursera.org/learn/neural-networks-deep-learning}{Neural Networks and 
    Deep Learning}~\cite{neural:coursera:nn}: Where the basic elements of a neural network and how
    to train it are explained.

  \item \href{https://www.coursera.org/learn/deep-neural-network}{Improving Deep Neural Networks: 
    Hyperparameter tuning, Regularization and Optimization}~\cite{neural:coursera:nn-hyperparameters}: 
    In this course it's shown the importance of hyperparameters and how each one works. 
    This way then it can be easier to design a proper network. Also, the different methods 
    of regularization are explained too, so overfitting can be avoided.

  \item \href{https://www.coursera.org/learn/convolutional-neural-networks}{Convolutional Neural 
    Networks}~\cite{neural:coursera:cnn}:
    How the \emph{convolution} operation works and why it's used in Machine Learning. 
    Different methods of using a Convolutional Neural Network, like face recognition or 
    object detection, are taught too.
\end{itemize}

All the tasks where done in three weeks.

\subsubsection{Get familiar with survival models like DeepSurv}

Survival Prediction models are a bit different from the typical Machine Learning problem. 
In this case, the desired output values are not the event indicator \( E \) and the time
interval \( T \). What we want to obtain is the survival function \( S(t) \) or the 
hazard function \( \lambda(t) \).

DeepSurv is one of the machine learning papers using a survival model, in this case the
Cox Proportional Hazards model which is not the same I will be planning to use.

To see how to properly use a survival model in a deep learning application i should fully 
understand how this is applied in the construction of the DeepSurv neural network. During 
the task I should compare the code implementation with the theoretical models so this way
I can see how to properly use \emph{vectorization} to speed-up computation
\cites{medical:cox}{medical:deep-surv}.

This process will take around two weeks.

\subsubsection{Preprocess data}

The input data for this project are:
\begin{itemize}
  \item RAW data from CT scans. There are around 88 slices for each patient, each one of 
  a size of \( 512 \times 512 \) pixels.
  \item Tumour annotations for the CT scans. For each RAW scan there's another slice which
  is a mask of 1s and 0s where the value is 1 if there's a tumour in that pixel.
  \item Clinical data. Which has information for each patient such as:
  \begin{itemize}
    \item Age
    \item Gender
    \item Smoking Pack Years
    \item Treatment
    \item Survival time
    \item Survival event
  \end{itemize}
\end{itemize}

The imaging data needs to be preprocessed so, for each CT scan, the tumour's bounding
box needs to be found and then extract the 3D slice. This slice needs to be resized 
normalized, to a value \( x \in [0, 1] \), and afterwards resized as a 3D image of 
\( 64 \times 64 \times 64 \) voxels. Additionally if data augmentation techniques are 
used then random crops and rotations to the image can be added before extracting the
3D slice. These operations can be don in the period of a week.

\subsubsection{Get familiar with Tensorflow}

Tensorflow\cite{neural:tensorflow} is one of the biggest libraries for developing deep neural 
networks. It hides from the user the underlying details of how to train a model, like how to 
compute de derivatives for each operation to be able to implement the back propagation 
algorithm. Also, it's open source so it comes at no cost.

Since it's a big piece of software and it offers many possibilities, it's very important to 
understand how it works and what is the best way to use it. The first step is to start by 
just being able to run the simple code that comes with the tutorials. Then, I should build
some simple models to test and see how to properly use the \texttt{Tensor} class that it
provides to create a computing graph and do the mathematical calculations.

This task won't take more than a week.

\subsubsection{Build shallow siamese network}

A siamese network is a type of neural network that is suited for comparison tasks such as 
face recognition or signature verification \cite{neural:siamese}.
Its design usually is made from two more networks, usually called sisters, then the output
of the two sisters gets joined in an extra layer which gives us the final output 
see \autoref{fig:siamese}. Usually the two sister networks share the same parameters and
architecture so after one is designed then it's duplicated and added the joining layer.

Through this approach the survival problem will be transformed into a order learning problem
where for two elements \( x_1 \) and \( x_2 \) the network learns the sorting function
\( f(x_1, x_2) = \hat{T}(x_1) > \hat{T}(x_2) \), where \( \hat{T}(x) \) is the predicted 
survival time.

Too build this model, a proper design for the sister network must be done first. This, will be
an iterative task where the proposed model will be created and then, trained optimizing
the hyperparameters. Afterwards, a new change will be made trying to get better results and
it will be compared against the previous one. 

The starting point can be seen in \autoref{fig:shallow-sister} which is inspired in a paper
\cite{neural:hand-eye-coordination} published by Google where they made a combination of 
scalar and image inputs in a Convolutional Neural Network.

Then the results will be compared and some few changes will be made to try to get better results.

\begin{figure}
  \centering
  \begin{tikzpicture}[node distance = .8]
  \tikzstyle{module}=[rounded corners, draw, inner sep = 5, text width=5em, align=center]
  \tikzstyle{sister}=[module, fill=cyan!20, text width=6em]
  \tikzstyle{out-node}=[module, fill=green!20]

  \node [sister] (S-A) at (0, 0) {Neural Network Architecture};
  \node [sister, below = 1.5 of S-A] (S-B) {Neural Network Architecture};
  
  \node [left = of S-A] (I-A) {Input A};
  \node [left = of S-B] (I-B) {Input B};
  
  \node [out-node, right = of S-A] (O-A) {Vector A \\ Output};
  \node [out-node, right = of S-B] (O-B) {Vector B \\ Output};
  \node (aux-1) at ($(O-A)!0.5!(O-B)$) {};

  \node [module, right = 1 of aux-1, fill=red!20] (D) {Distance \\ compare};

  \node [out-node, right = of D] (O) {Output \(\in (0, 1) \) };

  \draw [-latex] (I-A) -- (S-A);
  \draw [-latex] (I-B) -- (S-B);

  \draw [-latex] (S-A) -- (O-A);
  \draw [-latex] (S-B) -- (O-B);

  \draw[-latex] (O-A) -- (D);
  \draw[-latex] (O-B) -- (D);

  \draw [-latex] (D) -- (O);

  \draw [latex-latex] (S-A) -- (S-B) node[midway, left] {Weights};
\end{tikzpicture}
  \caption{Siamese Network basic structure \label{fig:siamese}}
\end{figure}

\begin{figure}
  \centering
  
\begin{tikzpicture}[node distance = .5]

  \tikzstyle{module}=[rounded corners, draw]
  
  \node [module] (CNN-1) at (0, 0) {Convolution};
  \node [below = .5 of CNN-1] (aux-1) {};
  \node [module, below = .5 of aux-1] (SF) {Spatial Filling};

  \node [circle, draw, right = of aux-1] (join) {\( + \)};

  \node [module, right = of join] (CNN-2) {Convolution};
  \node [module, right = of CNN-2] (FC-2) {FC};
  \node [right = of FC-2] (out) {To siamese};

  \node [left = 2 of CNN-1] (image) {Image};
  \node [module, left = of SF] (FC-1) {FC};
  \node [left = of FC-1] (scalar) {Scalar};

  \draw [-latex] (CNN-1) -- (join);
  \draw [-latex] (SF) -- (join);
  \draw [-latex] (join) -- (CNN-2);
  \draw [-latex] (CNN-2) -- (FC-2);
  \draw [-latex] (FC-2) -- (out);

  \draw [-latex] (image) -- (CNN-1);
  \draw [-latex] (scalar) -- (FC-1);
  \draw [-latex] (FC-1) -- (SF);
\end{tikzpicture}



  \caption{Shallow sister network design \label{fig:shallow-sister}}
\end{figure}

This is one of the most important parts of the project, since it will have a huge
impact in the final results. It will take around four weeks to complete and more or less
and it will be a trial and error process to try to see which architecture for the sister
network is best suited to solve the problem.

In this phase the designed networks will be very shallow to make small and fast improvements
since the training time could be an issue.

\subsubsection{Build deep siamese network}

Once an initial shallow network is already created then the next step will be to try 
and create a more deep version. To do so it will be done step by step adding one layer
at a time and seeing how the network is performing.

The main advantage of deep networks against shallow networks is that they are able to
identify more complex features and thus they usually perform better. However, its 
main drawback is that training time is longer and so the development process is 
slower.

\subsubsection{Compare the model against other ones}

Once the final deep model is created then it will be compared against different methods of 
survival analysis. Since this method will be the first one using both image data and scalar 
data it will be compared against models using only one or the other. 

To compare the model the cross-validation C-index will be obtained first. Afterwards,
it will be compared against a models trained using the same dataset but that use the
\emph{radiomics} features extracted with \texttt{PyRadiomics} from the CT scans but 
use a Cox Proportional Hazards model instead.
~\cites{medical:py-radiomics}{medical:cox}

To do all this comparisons the same original data, the head and neck dataset, must be 
used so it may require running again the training process using the compared method. 
That's why it will take two weeks for this task to complete.

\subsection{Estimated time}

In \autoref{tab:time} an estimation of the number of hours dedicated to each task is shown.

\begin{table}[H]
  \centering{}
  \begin{tabular}{|l|r|}
    \hline
    Task & Estimated duration (h) \\ \hline \hline
    Acquire background in CNN & 120 \\ \hline
    Get familiar with survival models & 80 \\ \hline
    Preprocess data & 40 \\ \hline
    Build shallow siamese network & 200 \\ \hline
    Build deep siamese network & 80 \\ \hline
    Compare models & 80 \\ \hline
    Final stage & 160 \\ 

  
    \hline \hline
    \textbf{Total} & 760 \\
    \hline
  \end{tabular}
  \caption{Estimated time for each task \label{tab:time}}
\end{table}

\subsection{Gantt chart}

\autoref{fig:gantt} shows the planning of the different tasks of the project in a Gantt chart.

\begin{figure}[H]
  \centering{}
  \def\gantttext{4cm}
  \begin{ganttchart}[
      time slot format=isodate,
      x unit = .9mm,
      y unit title = 0.7cm,
      y unit chart = 0.5cm,
      group label font = \tiny\bf,
      title label font = \tiny,
      bar label font = \tiny,
      vgrid={*{4}{draw=none},dotted,*{2}{draw=none}},
      hgrid,
      calendar week text = {\startday},
      link bulge = 2,
    ]{2018-02-14}{2018-06-29}
    \gantttitlecalendar{month=shortname, week} \\


    % Start gantt itself
    \ganttgroup{Preamble}{2018-02-14}{2018-03-25} \\
    \ganttbar{Getting started}{2018-02-14}{2018-02-18} \\
    \ganttlinkedbar{Learn about CNN}{2018-02-19}{2018-03-11} \\
    \ganttlinkedbar{Learn about DeepSurv}{2018-03-12}{2018-03-25} \\

    
    \ganttgroup{Build siamese network}{2018-03-26}{2018-05-20} \\
    \ganttbar{Preprocess data}{2018-03-26}{2018-04-01} \\
    \ganttlinkedbar{Build shallow network}{2018-04-02}{2018-05-06} \\
    \ganttlinkedbar{Build deep network}{2018-05-07}{2018-05-20} \\

    \ganttgroup{\parbox[r]{2.3cm}{Compare models and extract conclusions}}{2018-05-21}
    {2018-06-03} \\

    \ganttbar{Compare models}{2018-05-21}{2018-05-27} \\
    \ganttlinkedbar{Extract conclusions}{2018-05-28}{2018-06-03} \\

    \ganttgroup{Final Stage}{2018-06-04}{2018-06-29} \\
    \ganttbar{Write final report}{2018-06-04}{2018-06-17} \\
    \ganttbar{End project}{2018-06-18}{2018-06-24} \\
    \ganttbar{Presentation}{2018-06-25}{2018-06-29} \\

  \end{ganttchart}
  \caption{Gantt chart of the project \label{fig:gantt}}
\end{figure}

\subsection{Alternatives and action plan}

As it has already been said, the fact that there will be weekly meetings will allow to 
revise and adapt the initial planning. This means that, for each task, the real duration, 
compared to the planned one may differ, and thus the planning will be changed accordingly. 
If a task is finished before planned, then the next task will start immediately. 
Moreover, if a task duration is longer than expected, then the following tasks will need 
to be rescheduled or, in the worst case, omitted, although not all the tasks can be excluded.

Having weekly meetings with the Principal Investigator should mitigate the possible problems,
and give enough time to detect possible delays or differences with the original time, and
thus fix them as soon as possible.

Below, there are some examples of potential sources of delays.

\subsubsection{Complexity of the built model}

Since the model that it's going to be built is quite complex it will have a constant trial
and error. This can cause delays if a wrong architecture it's chosen, and it's not detected
in time, because a lot of time can be lost until the wrong choice is noticed. 

\subsubsection{Bugs}

Using complex software such as Tensorflow is prone to errors in many ways. Programming errors
are not an exception and while writing the code some bugs can be introduced. If a bug is not 
noticed soon it can cause long delays while trying to fix them.

\subsubsection{Unavailability of Compute Canada}

Since this project involves training Convolutional Neural Networks it needs a lot of computing 
power so a computing cluster will be used to train the data. If the cluster it's not available
then the training will need to be done with local resources which will be slower and thus 
more time will be lost.

\subsubsection{Training times}

The training time is one of the biggest drawbacks from this project. Since this task
can take a huge amount of time depending on the chosen architecture it can cause
huge delays from the original plan. This should be mitigated by doing small tests
instead of a whole training session to make sure that no bugs are found in the code 
because, otherwise, all the training results would become invalid.
