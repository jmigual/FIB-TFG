\section{Budget and sustainability}

\subsection{Project budget}

In order to carry out this project some resources will be needed. In this document, 
an estimation of the cost of the project is presented, keeping in mind the required
hardware and software resources, and its corresponding amortizations. Also, the
indirect costs of the project are counted too.

\subsubsection{Hardware budget}

In order to implement the different models, previously explained. Some hardware will be
needed. The training will usually be done in Compute Canada computing platform. the
access to the different provided clusters it's provided by the Bioinformatics and 
Computational Genomics Laboratory.

\autoref{tab:hardware-budget} shows an estimation of the cost of the hardware used in the
project, taking into account its useful life and its amortizations.

\begin{table}[H]
  \centering
  \begin{tabular}{|l|r|r|r|r|}
    \hline
    \textbf{Product} & \textbf{Price} & \textbf{Units} & \textbf{Useful life} 
    & \textbf{Amortization} \\ \hline\hline

    MSI GL62M 7RD-429XES & 1.000 € & 1 & 5 years & 83 € \\ \hline
    Lab iMac & 0 € & 1 & -- & 0 € \\ \hline
    Compute Canada platform & 0 € & 1 & -- & 0 € \\ \hline

    \hline\hline
    \textbf{Total} & 1.000 € & \multicolumn{2}{|c|}{} & 83 € \\ \hline
  \end{tabular}
  \caption{Hardware budget \label{tab:hardware-budget}}
\end{table}

\subsubsection{Software budget}

Some software will be used to carry out the project. Since this project will be used using
Open Source and free tools the total software cost of the project is zero. 
\autoref{tab:software-budget} shows an estimation of the cost of the software used in the 
project.

\begin{table}[H]
  \centering
  \begin{tabular}{|l|r|r|r|}
    \hline
    \textbf{Product} & \textbf{Price} & \textbf{Units} & \textbf{Amortization} \\ \hline\hline

    Python & 0 € & 1 & 0 € \\ \hline
    Tensorflow & 0 € & 1 & 0 € \\ \hline
    PyDicom & 0 € & 1 & 0 € \\ \hline
    NumPy & 0 € & 1 & 0 € \\ \hline
    Scikit & 0 € & 1 & 0 € \\ \hline
    Visual Studio Code & 0 € & 1 & 0 € \\ \hline
    Git & 0 € & 1 & 0 € \\ \hline
    GitHub & 0 € & 1 & 0 € \\ \hline
    PyCharm Community Edition & 0 € & 1 & 0 € \\ \hline
    \LaTeX & 0 € & 1 & 0 € \\ \hline
    Vim & 0 € & 1 & 0 € \\ \hline

    \hline\hline
    \textbf{Total} & 0 € &  & 0  € \\ \hline
  \end{tabular}
  \caption{Software budget \label{tab:software-budget}}
\end{table}


\subsubsection{Human resources budget}

To develop this software some roles will be taken. There will be the Project Manager role,
the Software Developer role and the Tester role. Each role is differentiated in the total
of the 760 hours. \autoref{tab:salary} shows the estimated costs for each role and the
Human Resources cost, \autoref{tab:time-estimation} shows an estimation of the time 
inverted on each role for each task.

\begin{table}[H]
  \centering
  \begin{tabular}{|l|r|r|r|}
    \hline
    \textbf{Role} & \textbf{Hours} & \textbf{€/hour} & \textbf{Salary} \\ \hline\hline

    Project Manager & 70 & 26 & 1.820 € \\ \hline
    Software Developer & 510 & 18 & 9.180 € \\ \hline
    Tester & 180 & 15 & 2.700 € \\ \hline

    \hline\hline 
    Total & 760 & & 13.700 € \\
    \hline
  \end{tabular}

  \caption{Human resources budget, salary according to \cite{ine:salary} \label{tab:salary}}
\end{table}

\begin{table}[H]
  \centering
  \begin{tabular}{|P{4cm}|r|r|r|r|}
    \hline
    \multirow{2}{*}[-1em]{\textbf{Task}} & 
    \multirow{2}{*}[-1em]{\textbf{Duration}} & 
    \multicolumn{3}{|c|}{\textbf{Dedication}} \\ \cline{3-5}

     & & \parbox[c][1.5cm]{2.1cm}{\textbf{Project \\ Manager}} & 
     \parbox[c][1.5cm]{2.2cm}{\textbf{Software \\ Developer}} & 
     \textbf{Tester} \\ \hline\hline

     Acquire background in CNN & 120 & 10 & 110 & 0 \\ \hline
     Get familiar with survival models & 80 & 10 & 70 & 0 \\ \hline
     Preprocess data & 40 & 10 & 20 & 10 \\ \hline
     Build shallow siamese network & 200 & 15 & 110 & 75 \\ \hline
     Build deep siamese network & 80 & 10 & 50 & 20 \\ \hline
     Compare models & 80 & 10 & 40 & 30 \\ \hline
     Final stage & 160 & 5 & 110 & 45 \\ 

     \hline\hline
     \textbf{Total} & 760 & 70 & 510 & 180 \\
     \hline
  \end{tabular}

  \caption{Time estimation by role \label{tab:time-estimation}}
\end{table}

\subsubsection{Unexpected costs}

As there may be unexpected changes in the project, an extra budget has to be added to compensate
for such problems. \autoref{tab:unexpected-costs} shows the distribution of this budget per role.

\begin{table}[H]
  \centering
  \begin{tabular}{|l|r|r|r|}
    \hline
    \textbf{Role} & \textbf{Hours} & \textbf{€ / hour} & \textbf{Salary} \\ \hline\hline
    Project Manager & 10 & 26 & 260 € \\ \hline
    Software Developer & 20 & 18 & 360 € \\ \hline
    Tester & 10 & 15 & 150 € \\ \hline

    \hline\hline
    \textbf{Total} & 40 & & 770 € \\ 
    \hline
  \end{tabular}
  \caption{Unexpected costs \label{tab:unexpected-costs}}
\end{table}

\subsubsection{Indirect costs}

In a project, there are some costs that cannot be added in neither of the previous categories,
this are estimated in \autoref{tab:indirect-costs}.

\begin{table}[H]
  \centering
  \begin{tabular}{|l|r|r|r|}
    \hline
    \textbf{Product} & \textbf{Price} & \textbf{Units} & \textbf{Cost} \\ \hline\hline

    Electricity & 0,12 €/kWh & 550 kWh & 66 € \\ \hline
    Internet + phone & 70 €/month & 4 months & 280 € \\ \hline
    
    \hline\hline
    \textbf{Total} & & & 346 € \\ \hline
  \end{tabular}
  \caption{Indirect costs \label{tab:indirect-costs}}
\end{table}

\subsubsection{Total budget}

By adding all the previously provided budgets, the total estimated budget for this project
is obtained, as shown in table \autoref{tab:total-costs}. Notice that a 5\% of contingency
has been added over the cumulative total in order to cover unexpected expenses that may 
occur during the development of the project.

\begin{table}[H]
  \centering
  \begin{tabular}{|l|r|}
    \hline
    \textbf{Concept} & \textbf{Estimated Costs} \\ \hline\hline

    Hardware resources & 83 € \\ \hline
    Software resources & 0 € \\ \hline
    Human resources & 13.700 € \\ \hline
    Unexpected costs & 770 € \\ \hline
    Indirect costs & 346 € \\ \hline

    \hline\hline
    \textbf{Subtotal} & 14.899 € \\
    \hline\hline
    Contingency (5\%) & 744.95 € \\
    \hline\hline
    \textbf{Total} & \textbf{15.643,95 €} \\ \hline
  \end{tabular}

  \caption{Total project costs \label{tab:total-costs}}
\end{table}

Finally, an estimation of the approximate budget per task is also provided in table 9. The
cost of each task is computed using the number of hours spent in the task and the required
resources.

\begin{table}[H]
  \centering
  \begin{tabular}{|l|r|}
    \hline
    \textbf{Task} & \textbf{Estimated cost} \\ 
    \hline\hline

    Acquire background in CNN & 2.436,04 € \\ \hline
    Get familiar with survival models & 1.653,03 € \\ \hline
    Preprocess data & 837,39 € \\ \hline
    Build shallow siamese network & 3.800,88 € \\ \hline
    Build deep siamese network & 1.587,78 € \\ \hline
    Compare models & 1.555,15 € \\ \hline
    Final stage & 3.028,74 € \\ 

    \hline\hline
    \textbf{Total} & 14.899,00 € \\ \hline
  \end{tabular}
\end{table}


\subsection{Budget monitoring}

In order to control the budget, at the end of each task the budget will be updated with the
effective amount of hours, the cost of the resources used and the expenses of the unexpected
events that may have occurred. Afterwards, these values will be compared against the previous
estimations to get indicators that show the amount of deviation from the initial planning.
The following formulas will be applied:

\[
  \text{Cost deviation} = (EC - RC) \cdot RH
\]
\[
  \text{Consumption deviation} = (EH - RH) \cdot EC
\]

\subsection{Sustainability and social commitment}

\subsubsection{Environmental dimension}

\subsubsection{Economical dimension}

\subsubsection{Social dimension}


