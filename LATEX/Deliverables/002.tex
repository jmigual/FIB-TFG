% !TEX root = main.tex


\section{Project Planning}

\subsection{Planning and scheduling}

The estimated project duration is of about 4 months. The project starts on Wednesday 14th of 
February, 2018 and the deadline is on Wednesday 20th June, 2018, the day before leaving the 
Benjamin Haibe-Kains Bioinformatics and Computational Genomics Laboratory.

During the development of the project there will be weekly lab meetings with all the
members of the laboratory where the development of the project of different lab members will
be discussed. Moreover, there will be a weekly meeting with prof. Benjamin Haibe-Kains to
discuss the work done and how to improve the project.

It must be noticed that the initial planning can be revised and updated as a result of the 
project's evolution and feedback received from the lab members. 

\subsection{Task description}

\subsubsection{Acquire background in Convolutional Neural Networks}

The first step is to acquire a better understanding in how a convolutional neural network works.
Therefore, in the las month I've been learning about Convolutional Neural Networks and how they
can be used. 
I started with basic statistics applied to \emph{Machine Learning} by reading the book 
\emph{The Elements of Statistical Learning}.~\cite{ElementsStatisticalLearning}

Then, I continued by doing three courses made by \href{https://www.deeplearning.ai}{Deeplearning.ai}
and published at Coursera~\cite{Coursera} related to Convolutional Neural Networks:
\begin{itemize}
  \item \href{https://www.coursera.org/learn/neural-networks-deep-learning}{Neural Networks and 
    Deep Learning}~\cite{Coursera:NN}: Where the basic elements of a neural network and how
    to train it are explained.

  \item \href{https://www.coursera.org/learn/deep-neural-network}{Improving Deep Neural Networks: 
    Hyperparameter tuning, Regularization and Optimization}~\cite{Coursera:NNHyperparameters}: 
    In this course it's shown the importance of hyperparameters and how each one works. 
    This way then it can be easier to design a proper network. Also, the different methods 
    of regularization are explained too, so overfitting can be avoided.

  \item \href{https://www.coursera.org/learn/convolutional-neural-networks}{Convolutional Neural 
    Networks}~\cite{Coursera:CNN}:
    How the \emph{convolution} operation works and why it's used in Machine Learning. 
    Different methods of using a Convolutional Neural Network, like face recognition or 
    object detection, are taught too.
\end{itemize}

\subsubsection{Get familiar with survival models and DeepSurv}

Survival Prediction models are a bit different from the typical Machine Learning model. This is
because, in this case, the loss function is not done by comparing the predicted values with 
the validation ones. DeepSurv is one of the machine learning papers using a survival model.

To see how to properly use a survival model in a deep learning application i should fully 
understand how this is applied in the construction of the DeepSurv neural network. During the task
I should compare the code implementation with the theoretical models
\cites{Cox}{DeepSurv}.
This process will take around two weeks.

\subsubsection{Get familiar with Tensorflow}

One week

\subsubsection{Create the regression model}

\subsubsection{Create the classification model}

\subsubsection{Compare the model against other ones}

\subsection{Estimated time}

In \autoref{tab:time} an estimation of the number of hours dedicated to each task is shown.

\begin{table}
  \centering{}
  \begin{tabular}{|l|r|}
    \hline
    Task & Estimated duration (h) \\ \hline \hline
    Acquire background in CNN & 70 \\ \hline
  
    \hline \hline
    \textbf{Total} & 450 \\
    \hline
  \end{tabular}
  \caption{Estimated time for each task \label{tab:time}}
\end{table}

\subsection{Gantt chart}

\autoref{fig:gantt} shows the planning of the different tasks of the project in a Gantt chart.

\begin{figure}
  \centering{}
  \def\gantttext{4cm}
  \begin{ganttchart}[
      time slot format=isodate,
      x unit = .9mm,
      y unit title = 0.7cm,
      y unit chart = 0.6cm,
      group label font = \tiny\bf,
      title label font = \tiny,
      bar label font = \tiny,
      vgrid,
      hgrid,
      calendar week text = {\startday},
      link bulge = 2,
    ]{2018-02-14}{2018-06-29}
    \gantttitlecalendar{year, month=shortname, week} \\


    % Start gantt itself
    \ganttgroup{Preamble}{2018-02-14}{2018-03-18} \\
    \ganttbar{Getting started}{2018-02-14}{2018-02-18} \\
    \ganttlinkedbar{Learn about CNN}{2018-02-19}{2018-03-09} \\
    \ganttlinkedbar{Learn about DeepSurv}{2018-03-10}{2018-03-16} \\

    \ganttgroup{Regression model}{2018-03-19}{2018-04-08} \\
    \ganttbar{Analysis}{2018-03-19}{2018-03-25} \\
    \ganttlinkedbar{Implementation}{2018-03-26}{2018-04-01} \\
    \ganttlinkedbar{Test}{2018-04-01}{2018-04-09} \\

    \ganttgroup{Classification model}{2018-04-09}{2018-04-29} \\

    \ganttgroup{Compare models}{2018-04-30}{2018-05-27} \\

    \ganttgroup{Final Stage}{2018-06-11}{2018-06-29} \\
    \ganttbar{End project}{2018-06-11}{2018-06-17} \\
    \ganttbar{Manual}{2018-06-18}{2018-06-24} \\
    \ganttbar{Presentation}{2018-06-25}{2018-06-29} \\

  \end{ganttchart}
  \caption{Gantt chart of the project \label{fig:gantt}}
\end{figure}

\subsection{Alternatives and action plan}
