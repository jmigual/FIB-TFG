% !TEX root = main.tex

\phantomsection
\addcontentsline{toc}{\seccn}{\numberline{}Notation}
\secc*{Notation}
\markboth{NOTATION}{NOTATION}

Through the thesis the following notation will be used:

\begin{table}[H]
  \centering
  \begin{tabular}{l|c|p{6cm}}
     & \textbf{Symbol} & \textbf{Description} \\
    \hhline{===}
    Scalars & \( x \) & \\
    Vectors & \( \bm{x} \) & \\
    Elements of vector & \( \bm{x}_i \) & \( i \)-th element of vector \( \bm{x} \) \\
    Matrices & \(\bm{X}\) & \\
    Desired values & \( \bar{x} \) & Desired value for \( x \) \\
    Predicted values & \( \hat{x} \) & Predicted value for \( x \) \\
    L2-norm & \( ||\bm{x}||_2 \) & \\
    \hline
    NN Layer & \( l \) & \\
    Number of layers & \( L \) & \\
    Units for layer & \( n^{[l]} \) & Units for layer \( l \) \\
    Weights & \( w_{ij}^{[l]} \) & Weight between unit \( i \) of layer \( l \) and unit
    \( j \) of layer \( l - 1 \) \\
    Weight matrix & \( \bm{W}^{[l]} \) & Weight 
    matrix between layers \( l \) and \( l - 1 \), 
    \( \bm{W}^{[l]} \in \mathbb{R}^{n^{[l]} \times n^{[l - 1]}} \). \\
    Activation & \( a_i^{[l]} \) & Activation of unit \( i \) for layer \( l \) \\
    Batch size & \( N \) & \\
  \end{tabular}
  \caption{Thesis' notation}
\end{table}

