\secc{Design and implementation}

\ssecc{Preprocess data}

As it was previously stated the data needs to be preprocessed before start training with it.
This steps are required because small changes can really help in reducing the time needed to
fit the model. By normalizing the data and setting variance to 1 and mean to 0 the training time
can be reduced in many epochs.

% TODO: Insert paper about batch normalization.

\sssecc{Image data}

The imaging data in the \Gls{PMHNK} contains XXX folders, each one with one patient. However, 
we only have the clinical data for XXY patients so the amount of data that we can use is
reduced. Moreover, because there was a problem with tumour annotations only 509 of the XXY 
patients have valid tumour annotations. This will be the final dataset size used for the
following training steps.




