\section{DeepSurv}

\begin{frame}{\insertsec}
    Paper published by Jared L.Katzman \emph{et al.} (2017) using a deep learning approach for
    the Survival Prediction problem.
    
    \begin{itemize}
        \item Shows an application of deep learning to survival analysis
        \item It uses this deep learning model to create a personalized treatment 
        recommender system
        \item Uses the Cox Proportional Hazards Model
    \end{itemize}
    
\end{frame}

\begin{frame}{The Survival Prediction Problem}
    Survival data has three elements: patient's baseline data \( x \), a failure event 
    time \( T \) and an event indicator \( E \).
    
    There's the survival function (\( S(t) = \Pr(T > t) \)) and the hazard function
    \( \lambda(t) \):
    \[
        \lambda(t) = \lim_{\delta \rightarrow 0} 
        \frac{\Pr(t \le T < t + \delta | T \ge t)}{\delta} 
    \]

    Given baseline data \( x \) the hazard function is assumed to have the form:
    \[
        \lambda(t | x) = \lambda_0 (t) \cdot e^{\bm{h(x)}}
    \]
\end{frame}

\begin{frame}{Linear Survival Models}
Cox Proportional Hazards model estimates that the risk function \( h(x) \) is linear 
\( \hat{h}_{\beta}(x) = \beta^T x \). The partial likelihood, parameterized by \( \beta \) is 
defined:

\[
    L_c (\beta) = \prod_{i:E_i = 1} \frac{\exp(\hat{h}_{\beta}(x_i))}
    {\sum_{j \in \mathfrak{R}(T_i)} \exp(\hat{h}_{\beta}(x_j))}
\]
\[
    \mathfrak{R}(t) = \{i: T_i \ge t\}
\]
    
\end{frame}

\begin{frame}{Nonlinear survival models}
The used loss function is the following one
    
\[
    l(\theta) := -\sum_{i:E_i = 1} \left(
    \hat{h}_{\theta}(x_i) - \log \sum_{j \in \mathfrak{R}(T_i)} e^{\hat{h}_{\theta} (x_j)}
    \right)
\]
\end{frame}